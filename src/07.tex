\documentclass[__main__.tex]{subfiles}

\begin{document}

\section{Метод Галеркина. Лемма Сеа. Явный вид СЛАУ, к которой приводит метод. Лемма об условиях сходимости метода к точному решению.}

\begin{definition}[Метод Галёркина]
	Пусть $H_{(.)}$ - последовательность замкнутых подпространств в $H.$
	Тогда приближённым решением по методу Галёркина будет:
	\begin{gather*}
	U_n \in H_n,\\
	\forall V_n \in H_n: B(U_n,V_n)=f(V_n).
	\end{gather*}
\end{definition}

\begin{theorem}[Лемма Сеа]
	Пусть $U_n$ - приближённое решение по методу Галёркина. Тогда
	\begin{gather*}
	\parallel U-U_n\parallel\le\frac{M}{m}\parallel U-V_n\parallel\;\forall V_n\in U_n.
	\end{gather*}
\end{theorem}

\begin{proof}
\begin{gather*}
B(U-U_n ,U-U_n)=B(U-U_n ,U-V_n)+B(U-U_n ,V_n-U_n)=\\
B(U-U_n ,U-V_n)+\underbrace{B(U,V_n-U_n)}_{f(V_n-U_n)}-\underbrace{B(U_n,V_n-U_n)}_{f(V_n-U_n)}=
B(U-U_n,U-V_n).\\
m\parallel U-U_n\parallel^2\le B(U-U_n,U-U_n)=B(U-U_n,U-V_n)\le M\parallel U-U_n\parallel\parallel U-V_n\parallel
\end{gather*}
\end{proof}
	\begin{theorem}[Лемма о сходимости ( следствие из Сеа )]Пусть $H_{(.)}$ предельно плотно в $H$. Тогда последовательность $U_{(.)}$ по методу Галёркина сходится к точному решению задачи.
\end{theorem}
	Рассмотрим пример метода Галёркина, который мне удалось найти, а именно решение дифференциального ур-я методом КЭ.\\
	$C^n_0[a,b]=\left\{x\in C^n[a,b]: x(a)=x(b)=0\right\}.$\\
	Пусть $p\in C^1[a,b];\;q,f\in C[a,b];\;p(\tau)\ge c_0=const;\;q(\tau)\ge 0\;\forall\tau\in[a,b];\;U\in C^2_0[a,b]$ и
	\begin{gather*}
	\begin{cases}
	\mathbf{L}U=-\frac{d}{d\tau}\left(p\frac{dU}{d\tau}\right)+qU=f,\\
	U(a)=U(b)=0
	\end{cases}.
	\end{gather*}
	
	Берём произвольную ф-ю $V\in C^1_0[a,b]$, домножаем и интегрируем:
	\begin{gather*}
-\int\limits_a^b V\left[\frac{d}{d\tau}\left(p\frac{dU}{d\tau}\right)+qU\right]d\tau=
\underbrace{\int\limits_a^b\left(pU\prime V\prime+qUV\right)d\tau}_{B(U,V)}
	=\underbrace{\int\limits_a^b Vfd\tau}_{f(V)}\Rightarrow\\
	B(U,V)=f(V)\;\forall V\in C^1_0[a,b]\Rightarrow
	\int\limits_a^b\left(\mathbf{L}U-f\right)Vd\tau=0.\\
	Let\;U_n(\tau)=\sum\limits_{i=1}^{n-1}c_i h_i(\tau),\;\Rightarrow
	B(U_n,V)=\sum\limits_{i=1}^{n-1}c_i B(h_i,V)=f(V).\\
	Let\;V=h_j,\;\Rightarrow\sum\limits_{i=1}^{n-1}c_i \underbrace{B(h_i,h_j)}_{K^i_j}=\underbrace{f(h_j)}_{f}
	\end{gather*}
	Тогда получим явный вид СЛАУ: $Kc=f$, a $h_j$ - например линейный сплайны.
\end{document}
