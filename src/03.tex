\documentclass[__main__.tex]{subfiles}

\begin{document}

\section{Фундаментальная последовательность в метрическом пространстве. Полное метрическое пространство. Гильбертово пространство. Уравнение с липшицевым сильно монотонным оператором в гильбертовом пространстве. Теорема существования и единственности решения.}

\begin{definition}
    Метрическое пространство.\\
    $X$ - некоторое множество с заданым на нем $p: X \times X \Rightarrow R $, которое обладает свойствами:\\
    \begin{enumerate}
        \item $p(x,y) \geq 0 : p(x,y) = 0 \leftrightarrow x=y$
        \item $p(x,y) = p(y,x)$
        \item $p(x,y) \leq p(x,z) + p(z,y)$
    \end{enumerate}
    Пример.\\
    $p(x,y) = \sqrt{\sum_{i=1}^n (x_i - y_i)^2}$
\end{definition}

\begin{definition}
    Фундаментальная последовательность в метрическом пространстве.\\
    Пусть $X$ - метрическое пространство\\
    $a_n \ \forall \varepsilon > 0 : \exists N(\varepsilon) : \forall m,n > N(\varepsilon) \ p(a_n,a_m)< \varepsilon$
\end{definition}

\begin{definition}
    Полное метрическое пространство.\\
    Метрическое пространство $(x,p)$ называется полным, если всякая фундаментальная последовательность сходится к элементу $X$.
\end{definition}

\begin{definition}
    Гильбертово пространство.\\
    Гильбертово пространство - банохово пространство, норма которого порождена положительно определенным скалярным произведением.
\end{definition}

\textbf{Теорема.}\\
Пусть $A: H \Rightarrow H$ сильно монотонный липшицев оператор в гильбертовом пространстве $H$. Тогда $\forall f \in H \exists ! x \in H$.\\
$Ax = f$
\begin{proof}
    Пусть $\tau \neq 0, S_{\tau} \circ x = x - \tau(Ax-f)$\\
    $Ax = f \Leftrightarrow S_{\tau} \circ x = x$\\
    $\rho^2 (S_{\tau} \circ x, S_{\tau} \circ y) = ||S_{\tau}x - S_{\tau}y||^2 = ||x - y - \tau(Ax - Ay)||^2 = ||x-y||^2 - 2\tau (Ax - Ay,x-y) + \tau^2 ||Ax - Ay||^2 \leq ||x - y||^2 - 2\tau m ||x-y||^2 + \tau^2 M^2 ||x-y||^2 = (1-2\tau m + \tau^2 M^2) \rho^2 (x,y)$\\
    Подберем $\tau$, чтобы было $<1$ и $>0$\\
    $0 < 1-2\tau m + \tau^2 M^2 <1$\\
    $\tau \in (0,\frac{2m}{M^2})$\\
    $\rho (S_{\tau} \circ x, S_{\tau} \circ y) \leq L(\tau) \rho(x,y)$\\
    $0<L(\tau)<1$
\end{proof}


\end{document}
