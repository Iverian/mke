\documentclass[__main__.tex]{subfiles}

\newcommand{\norm}[1]{\left\Vert{#1}\right\Vert}


\begin{document}

\section{Краевая задача для стационарного уравнения теплопроводности с граничными условиями первого и второго рода. Классическая постановка задачи. Вывод вариационного уравнения. Определение слабого решения задачи. Достаточное условие существования классического решения задачи. Теорема существования и единственности слабого решения задачи.}

Пусть \(\Omega\subset\mathbb{R}^3\), \(Ox_i\) -- декартова СК с ортами \(\symbf{e}_i\), \(i\in\overline{1\dots3}\) и \(f=\mathring{p}q_m{\in}C(\Omega)\). Представим границу как \(\partial{\Omega}=\Sigma_{q}\cup\Sigma_{\theta}\), где \(\Sigma_{q}\cap\Sigma_{\theta}=\phi\).

\(\tilde{\lambda}=\lambda_{ij}\symbf{e}_i\otimes\symbf{e}_j\), \(\lambda_{ij}{\in}C(\Omega)\).

Требуется найти температуру \(\theta\in{C^2(\Omega)}\cap{C(\bar\Omega)}\) такую, что:

\begin{gather*}
	\left\{
	\begin{gathered}
		\nabla\symbf{q}=\mathring{p}q_m, \quad \forall\tau\in\Omega\hfill\\
		\symbf{q}=-\tilde{\lambda}\cdot\nabla\theta\hfill\\
		\left.\symbf{q}\cdot\symbf{n}\right|_{\Sigma_q}=q_e\hfill\\
		\left.\theta\right|_{\Sigma_{\theta}}=\theta_{e}\hfill\\
	\end{gathered}
	\right.{.}
\end{gather*}

Рассмотрим \(v\in{C^1(\bar\Omega)}\), где \(\left.v\right|_{\Sigma_\theta}=0\):

\begin{flalign*}
	&
	\int\limits_{\Omega}v\nabla\cdot\symbf{q}{d\tau}
	=
	\int\limits_{\Omega}vf{d\tau},\\
	&
	\nabla(v\symbf{q})
	=
	v\nabla\symbf{q}
	+
	\symbf{q}\nabla{v}\Rightarrow v\nabla\symbf{q}
	=
	\nabla(v\cdot\symbf{q})-\symbf{q}\nabla{v},\\
	&
	\int\limits_{\Omega}v\nabla\symbf{q}{d\tau}
	=
	\int\limits_{\Omega}\nabla(v\cdot\symbf{q}){d\tau}
	-
	\int\limits_{\Omega}\symbf{q}\cdot\nabla{v}{d\tau}
	=
	\int\limits_{\Sigma_q}vq_e
	+
	\int\limits_{\Omega}\nabla\theta\cdot\tilde\lambda{d\tau},\\
	&
	\underbrace{\int\limits_{\Omega}\nabla\theta\cdot\tilde\lambda{d\tau}}_{B(\theta,v)}
	=
	-\underbrace{\int\limits_{\Sigma_q}vq_e{d\tau}+\int\limits_{\Omega}vf{d\tau}}_{f(v)},
\end{flalign*}

получим вариационное уравнение $B(\theta,v)=-f(v)$.

Пусть \(f{\in}L_2(\Omega)\), \(q_e{\in}L_2(\Sigma_q)\), \(\theta_e{\in}L_2(\Sigma_\theta)\) и \(\forall\tau\in\Omega\colon\left|\lambda_{ij}(\tau)\right|{\le}m\), где \(m=\mathrm{const}\). Тогда $\lambda_{ij}$ -- интегрируема по Лебегу и \(\forall\symbf{x}\in\mathbb{R}^3\colon\symbf{x}\cdot\tilde{\lambda}\cdot\symbf{x}{\ge}m\symbf{x}\cdot\symbf{x}\forall\tau\in\Omega\).

\(V=\left\{v{\in}H^1(\Omega)\colon{TR}_{\Sigma_\theta}{\circ}v{\le}0\right\}\).

Пусть \(\omega{\in}H^1(\Omega)\) -- такая функция, что \({TR}_{\Sigma_\theta}\circ\omega=\theta_{e}\)

\begin{definition}
	Слабым решением стационарной краевой задачи теплопроводности будем называть такой элемент \(\theta{\in}H^1(\Omega)\), что \({\forall}v{\in}V{\colon}B(\theta,v)=f(v)\) и \(\theta-\omega{\in}V\).
\end{definition}

\begin{theorem}
	Пусть \(\Omega\) -- область с липшицевой границей и выполняются уравнения на входные данные задачи, тогда слабое решение \(\theta{\in}H^1(\Omega)\) существует и единственно, независимо от выбора \(\omega\).
\end{theorem}
\begin{proof}
	Пусть \(v{\in}V\), тогда:
	\begin{gather*}
		B(u,v)=\int\limits_{\Omega}\nabla{u}\cdot\tilde{\lambda}\nabla{v}{d\tau}{\ge}m\sum_{i=1}^{3}\left(\frac{\partial{v}}{\partial{x_i}}\right)^2{d\tau}{\ge}\left.{C\norm{v}^2}\right|_{H^1(\Omega)},
	\end{gather*}
	следовательно, \(B\) положительно определена в $V$. Пусть $u,v{\in}H^1(\Omega)$:
	\begin{flalign*}
		B(u,v)^2
		&=
		\left({\int\limits_{\Omega}\nabla{u}\tilde{\lambda}\cdot\nabla{v}{d\tau}}\right)^2
		=
		\left(\nabla{u},\tilde{\lambda}\right)^2_{[H^1(\Omega)]^3}
		\le\\
		&\le
		\norm{\nabla{u}}^2_{[H^1(\Omega)]^3}
		\cdot
		\norm{\tilde{\Lambda}\nabla{v}}^2_{[H^1(\Omega)]^3},
		\\
		\norm{\nabla{u}}^2_{[H^1(\Omega)]^3}
		&=
		\sum_{i=1}^{3}\int\limits_{\Omega}\left(\frac{\partial{u}}{\partial{x_i}}\right)^2{d\tau}
		\le
		\norm{u}_{H^1(\Omega)}
		\\
		\bar{q}
		&=
		\tilde{\lambda}\cdot\nabla{v},
		\\
		\norm{q}^2_{[H^1(\Omega)]^3}
		&=
		\sum_{i=1}^{3}\int\limits_{\Omega}q^2_i{d\tau}
		\le
		M\sum_{i=1}^{3}\int\limits_{\Omega}\left(\sum_{j=1}^{3}\frac{\partial{v}}{\partial{x_i}}\right)^2{d\tau}
		\le\\
		&\le
		{\xi}M\sum_{i,j=1}^{3}\int\limits_{\Omega}\left(\frac{\partial{v}}{\partial{x_j}}\right)^2{d\tau}
		\le
		\tilde{M}\sum_{j=1}^{3}\int\limits_{\Omega}\left(\frac{\partial{v}}{\partial{x_i}}\right)^2{d\tau}
		\le
		{\tilde{M}\norm{v}^2}_{H^1(\Omega)},
	\end{flalign*}
	т.е. \(B(u,v)^2{\le}\tilde{M}\norm{u}_{H^1(\Omega)}\norm{v}^2_{H^1(\Omega)}\) \(\Rightarrow\) \(B(u,v)\) -- непрерывна в \(H^1(\Omega){\times}H^1(\Omega)\) и положительна определена в $V{\times}V$, тогда по лемме Лакса-Мильграмма \(\exists{!}z{\in}V\), что \({\forall}v{\in}V{\colon}B(z,v)=f(v)-B(\omega,v),B(z+\omega,v)=f(v)\).

	Заметив, что \(u=z+w{\in}H^1(\Omega)\), получим \(u-\omega{\in}V\):
	\begin{gather*}
		{\forall}v{\in}V{\colon}B(u,v)=f(v)
	\end{gather*}

	Осталось показать единственность решения: пусть \({\exists}u_1,u_2{\in}H^1(\Omega){\colon}B(u_1,v)=f(v){\forall}v{\in}V{\colon}u_1-\omega_1{\in}V\text{ и }B(u_2,v)=f(v){\forall}v{\in}V{\colon}u_2-\omega_2{\in}V\). Тогда:
	\begin{gather*}
		u_1-u_2
		=
		\left[(u_1-\omega_1)-(u_2-\omega_2)+(\omega_1-\omega_2)\right]{\in}V
		\\
		0
		=
		B(u_2,v)-B(u_1,v)
		\\
		{\forall}v{\in}V{\colon}B(u_2-u_1,v)=0 \quad\text{пусть }v=u_2-u_1
		\\
		0
		=
		B(u_2-u_1,u_2-u_1){\ge}m\norm{u_2-u_1}^2_{H^1(\Omega)}{\Rightarrow}u_1=u_2
	\end{gather*}
\end{proof}

\end{document}
