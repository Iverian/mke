\documentclass[__main__.tex]{subfiles}

\begin{document}

\section{Уравнение с ограниченным самосопряженным оператором. Теорема об эквивалентности постановок задач в форме минимизации квадратичного функционала (функционала Ритца) и в форме вариационного уравнения.}

\subsection{Функционал Ритца}
\begin{definition}
	Пусть  оператор А действует из E в F, и функционал $\phi$ принадлежит $F^{*}$. Рассмотрим $f(x)=\phi(Ax), |\phi(x)|\leq||\phi \cdot ||A|| \cdot ||x||.$ Получили новый функционал f, принадлежащий  $E^{*}. \phi \rightarrow \phi A. $ \
	$\phi A = A^{*}:F^{*}\rightarrow E^{*}. A^{*}$ - \textbf{сопряженный оператор} к А. \\
	 Оператор $A=A^{*}$ - \textbf{ограниченный}
\end{definition}
\begin{definition}
Пусть B - симметричная положительно опредленная билинейная форма. Тогда $F:\mathbb{H}\rightarrow \mathbb{R}$:
$F(x)=\frac{1}{2}\mathbb{B}(x,x)-f(x)$ - \textbf{Функционал Ритца}
\end{definition}
\subsection{Теорема об эквивалентности постановок задач}
\begin{theorem}
	Пусть B - симметричная положительно определенная билинейная форма, тогда существует один и только один $x\in \mathbb{H}$, что $F(v)\rightarrow min$, причем $\forall v\in \mathbb{H}    B(x,v)=f(v)$
	\\
	\textbf{Док-во:}
	\\
	$[x,y]_{A}=B(x,y)$ - скалярное произведение в $\mathbb{H}$.  $||x||^{2}=[x,x]_{A}$
	$$||x-y||^{2}=||x||^{2}+||y||^{2}-2[x,y]_{A} \Rightarrow -[x,y]_{A}=\frac{1}{2}||x-y||^{2}_{A}-\frac{1}{2}||x||^{2}_{A}-\frac{1}{2}||y||^{2}_{A}
	$$
	$$F(v)=\frac{1}{2}||v||^{2}_{A}-f(v)=\frac{1}{2}||v||^{2}_{A}-[x,v]_{A}=\frac{1}{2}||x-v||^{2}_{A}-\frac{1}{2}||x||^{2}_{A} \geq -\frac{1}{2}||x||^{2}_{A}=F(x)$$
	
\end{theorem}
\end{document}
