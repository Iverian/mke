\documentclass[__main__.tex]{subfiles}

\begin{document}
	
	\section{Метод Ритца. Теорема об оценке точности приближенного решения задачи по методу Ритца. Лемма об условиях сходимости метода к точному решению. Явный вид СЛАУ, к которой приводит метод.}
	
    Элемент $u_n \in H_n$ называется приближенным решением задачи $Au = f$(f - заданный элемент) по методу Ритца , если 
    $$F(u_n)=min(F(v)) v\in H_n$$
	Теорема об оценке точности приближенного решения задачи по методу Ритца. 
	
	Пусть u $\in$ H:
	$$B(u,v)= f(v) \forall v\in H$$
	u(0) - приближения по методу Ритца относительно H(0)
	Тогда :
	$$||u-u_n|| \le \sqrt{\frac{M}{m}}*\delta_n (u)$$
где
$$\delta_n (u) = inf ||u-v_n||$$
Доказательство:
$$F(u)=F_0 (v)-\frac{1}{2} ||u||^2 _A$$
$$F_0(v)=\frac{1}{2} ||u-v||^2 _A$$
$$argminF(v)=argminF_0 (v)=u_n$$
$$F_0 (u_n)=\frac{1}{2}||u-u_n||{^2}{_A}\le F_0 (v_n)=\frac{1}{2}||u-v_n||{^2}{_A} $$
$$m||u-u_n||{^2}\le ||u-u_n||{^2}{_A}\le M||u-v_n||^2$$
$$||u-u_n||\le  \sqrt{\frac{M}{m}}||u-v_n|| => ||u-u_n|| \le \sqrt{\frac{M}{m}}*\delta_n $$
$$
\delta_n = inf ||u-v_n|| 
$$
где u - точное решение задачи
Пусть семейство подпространств $H_n$, n=1,2... предельно полно, т.е. $$\delta_n(v)=inf||v-v_n||\rightarrow 0 n \rightarrow \inf \forall v\in H$$
Тогда последовательность $u_n$ приближений по методу Ритца сходится к u.
\end{document}