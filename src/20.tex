\documentclass[__main__.tex]{subfiles}

\begin{document}

\section{Компактное хранение разреженных матриц. Портрет разреженной матрицы. CSR-формат представления несимметричной матрицы. 3 Модификация формата CSR для матриц с симметричным портретом. Алгоритм матрично-векторного умножения. CSlR-формат представления симметричной матрицы. Методы Крыловского типа для решения СЛАУ. Метод сопряженных градиентов.}

Пусть $A = \left(A^i_j\right)^n_n$, тогда \textbf{портретом матрицы} $A$ называется множество $P_a = \left(\left(i, j\right): a_{ij} \neq 0\right)$.

Допустим варианты:

\begin{enumerate}
	\item Матрицы с несимметричным портретом, то есть $\exists \left(i, j\right) \in P_A: \left(j, i\right) \notin P_A$. Тогда $A$ - несимметрична (то есть $A^T \neq A$).
	
	\item Неимметричные матрицы с симметричным портретом, то есть $\left(i, j\right) \in P_A \Leftrightarrow \left(j, i\right) \in P_A$.
	
	\item Симметричная матрица: $A^T = A$.
\end{enumerate}

CSR - формат представления разрешимой матрицы. Для этого вводятся следующие массивы:

\begin{itemize}
	\item $aelem$ - строчная запись ненулевых элементов матрицы
	
	\item $jptr$ - имеет то же число элементов, что и $aelem$. При этом $jptr\left[i\right]$ = номер столбца элемента матрицы $aelem\left[i\right]$.
	
	\item $iptr$ - имеет число элементов равное размерности СЛАУ + 1 (число строк матрицы А + 1). Причём $iptr\left[i\right]$ - с какой позиции в массиве $aelem$ и $jptr$ начинается $i$ - я строка А. То есть $iptr\left[i + 1\right] - iptr\left[i\right]$ - число ненулевых элементов в $i$ - й строке А.
\end{itemize}

Модификация для матриц с симметричным портретом:

\begin{itemize}
	\item Диагональ в массиве $adiag$
	
	\item Вместо $aelem$ используется $altr$, в котором хранятся только поддиагональные элементы, для которых формулируются $iptr, jptr$
	
	\item для поддиагональных элементов формулируется массив $autr$, в котором элементы записываются в столбцовом формате
\end{itemize}

Формат CSIR для симметричных матриц. В этом формате используется массив $autr$.

Матрично - векторные операции: $z = Ax, x \in \mathbb{R}^n$.

\begin{enumerate}
	\item Для CSR формата:
	\begin{lstlisting}
	for (i = 0 ... n - 1){
		z[i] = 0
		for (j = iptr[i] ... iptr[i+1] - 1) {
			z[i] += x[jptr[j]]aelem[j];
		}
	}
	\end{lstlisting}
	
	\item Модификация CSR для симметричного портрета
	\begin{lstlisting}
	for (i = 0 ... n-1) {
		z[i] = x[i] adiag[i]
	}
	for (i = 0 ... n-1) {
		for (j = iptr[i] ... iptr[i+1] - 1) {
			z[i] += x[jptr[j]] altr[j]
			z[jptr[j]] += x[i] autr[j]
		}
	} 
	\end{lstlisting}
	
	\item CSIR - то же, что и в предыдущем, с учётом, что $autr = altr$
\end{enumerate}

Про методы Крыловского типа. Задаётся подпространство $K, L \subset \mathbb{R}^n$. Требуется найти такой $x \in K$ , что $\left(Ax, l\right) > \left(b, l\right), \forall l \in L $.

Существует $x_0$ - начальное приближение решения x, тогда $\delta x$ - уточняющая поправка, то есть $x_1 = x_0 + \delta x$.

$$
\left(\left(b - A x_0\right) - A \delta x, l\right) = \left(r_0 - A\delta x, l\right) = 0, \forall l \in L
$$
где $r_0$ - невязка начального приближения.

Метод сопряжённых градиентов. Пусть дана СЛАУ $Ax = b$, причём матрица система - это действительная, симметричная и положительно определённая  матрица. Тогда процесс решения СЛАУ можно представить как минимизацию функционала: $$\left(Ax, x\right) - 2 \left(b,x\right) \rightarrow min.$$ Алгоритм:

\begin{enumerate}
	\item Выбираем начальное приближение $x^0$
	
	\item $r^0 = b - Ax^0$
	
	\item $z^0 = r^0$
	
	\item $\alpha_k = \frac{\left(r^{k-1}, r^{k - 1}\right)}{\left(Az^{k-1}, z^{k-1}\right)}$
	
	\item $x^k = x^{k-1} + \alpha_k z^{k-1}$
	
	\item $r^k = r^{k-1} - \alpha_k A z^K-1$
	
	\item $\beta_k = \frac{\left(r^k, r^k\right)}{\left(r^{k-1}, r^{k-1}\right)}$
	
	\item $z^k = r^k + \beta z^{k-1}$
\end{enumerate}

Критерий остановки - относительная $\left(\frac{\| r^k \|}{\| b \|}\right)$ или абсолютная $\left(\| r^k \|\right)$
\end{document}
